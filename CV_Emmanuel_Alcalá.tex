\documentclass{resume} % Use the custom resume.cls style

\usepackage[left=0.75in,top=0.6in,right=0.75in,bottom=0.6in]{geometry} % Document margins
\usepackage{fontawesome}
\usepackage{hyperref}
\hypersetup{
    colorlinks=true,
    linkcolor=blue,
    filecolor=magenta,      
    urlcolor=cyan,
}
\usepackage{graphicx}
% Skill bars filled proportionally
\usepackage{xcolor}
\usepackage{fancyhdr}
\usepackage{array,longtable,picture}
\definecolor{noskillgray}{gray}{0.85}
\definecolor{skilledblue}{rgb}{0.05,0.05,0.65}

\makeatletter
\newdimen\skillb@level
\newdimen\skillb@length
\newdimen\skillb@height
\skillb@length=120pt%
\skillb@height=10pt%
\newcommand*{\skillbar}[1]{%
	\skillb@level=\dimexpr#1\skillb@length/100\relax%
	{\color{skilledblue}\rule{\skillb@level}{\skillb@height}}%
	{\color{noskillgray}%
		\rule{\dimexpr\skillb@length-\skillb@level\relax}{\skillb@height}}%
}
\newcommand*{\skill}[2]{%
	\par\noindent%
	{\hskip 1ex\small #1}\\%
	\skillbar{#2}%
}
\makeatother

% -----------------------------------------------------------------------------
% \noindent % 4cm is the picture's width, -6cm by trial and error

% \begin{picture}(0,0)
% \put(\dimexpr\textwidth-4cm,-6cm){\includegraphics[width=4cm]{pic_CV.jpeg}}
% \end{picture}

\name{Emmanuel Alcalá} % Your name

%\address{123 Broadway \\ City, State 12345} % Your address
%\address{123 Pleasant Lane \\ City, State 12345} % Your secondary addess (optional)
%\address{WebPage: \url{https://jealcalat.github.io/} \tiny{\faExternalLink}} % Your phone number and email
\address{\faMobile \hspace{1ex} {\sf 33 14 99 93 16}}

% \address{e-mail:}
\address{ \faEnvelope \hspace*{0.2em}%
                 \texttt{jealcalat@gmail.com} \\
                 \texttt{jaime.alcala@iteso.mx}}
                 
\address{\faGithub \hspace{1ex} \url{https://github.com/jealcalat}}

\begin{document}
% \thispagestyle{fancy}

% \makecvheader

\begin{picture}(0,0)
    \put(0,-9){\fbox{\includegraphics[width=10em]{pic_CV.jpeg}}}
\end{picture}

% \begin{flushright}
% \includegraphics[scale=0.05]{pic_CV.jpeg}
% \end{flushright}

\begin{rSection}{Perfil}
    
    %Tengo una formación ecléctica que me ha facilitado desempañarme en diversas áreas, tanto educativas como de investigación. En maestría modelé computacionalmente un fenómeno de comportamiento económico de decisión intertemporal usando redes neurales. En el doctorado trabajé con modelos matemáticos en comportamiento de fenómenos psicológicos y económicos (elección intertemporal, formación de hábitos, etc.), así como la adecuación estadística de dichos modelos. También, implementé algoritmos de visión por computadora y \textit{posture tracking} (como \href{http://www.mackenziemathislab.org/deeplabcut}{DeepLabCut}). En todos los casos se trata de proyectos interdisciplinarios en los que he usado herramientas matemáticas, estadísticas y computacionales para modelar fenómenos diferentes. 
    Tengo una formación ecléctica que me ha facilitado desempañarme en diversas áreas, tanto educativas como de investigación. En maestría y en doctorado modelé computacionalmente fenómenos de comportamiento económico (toma de decisiones intertemporales y bajo incertidumbre) usando modelos de redes neurales, modelos generativos y probabilísticos. Uno de mis intereses durante ese período fue el uso de IA para estudiar cuantitativamente fenómenos económicos y psicobiológicos, para conocer cómo los seres vivos aprenden estructuras ricas de información imperfecta y bajo incertidumbre. También implementé modelos de visión por computadora, como OpenCV y \textit{posture tracking} (\href{http://www.mackenziemathislab.org/deeplabcut}{DeepLabCut}). Actualmente, me encuentro inmerso en varios proyectos, uno de los cuales requiere del uso Deep Learning para estudiar los trazos y posturas durante la escritura de niños en edad preescolar. En ITESO realicé estancia posdoctoral, y también he impartido las materias de Decisiones y Teoría de Juegos y Econometría. En mi estancia posdoctoral gané experiencia en la implementación de diseños de experimentos, análisis de regresión y en la metodología de superficies de respuesta. Desde el 2022 soy miembro del Sistema Nacional de Investigadores, nivel Candidato.  %Uno de mis intereses es el uso de IA para estudiar cuantitativamente fenómenos económicos y psicobiológicos. Otro es conocer cómo los seres vivos aprenden estructuras ricas de información imperfecta y bajo incertidumbre. 
    
\end{rSection}

\begin{rSection}{Educación}

{\bf Universidad de Guadalajara, CUCIénega} \hfill {\em 2008 - 2012} \\ 
Lic. Químico Farmacobiólogo \\
% Tesis: OGM y estandarización de Western-Blot \\
{\bf Universidad de Guadalajara, CEIC-CUCBA} \hfill {\em 2015 - 2017} \\ 
Maestría en Ciencias de la Conducta \\
Fecha de examen: 6 de julio de 2017\\
Tesis: Modelo de redes neurales de elección impulsiva \\
{\bf Universidad de Guadalajara, CEIC-CUCBA} \hfill {\em 2017 - 2021} \\
Doctor en Ciencias de la Conducta \\
Fecha de examen: 24 de mayo de 2021\\
Tesis: Formación de hábitos y resistencia al cambio; modelos computacionales

\end{rSection}

\begin{rSection}{Premios y distinciones}

{\bf Miembro de Sistema Nacional de Investigadores - C} \hfill {\em 1 enero 2022 - }\\
Vigencia: 1 enero del 2022 a 31 de diciembre de 2025

\end{rSection}

\begin{rSection}{Publicaciones}

	{\em 2018} \\
	{\bf Alcalá, E.}, \& Arámbula-Román, J.C.(2018). El consumidor contra la democracia, y por qué retomar la psicología no reduccionista. En \textit{Capital social, decentralización y participación ciudadana: entre la reflexión y la evidencia}. (A.R. Cogco-Calderón \& Pérez-Cruz, J.A, Eds). Ciudad de México: UAT-Colofón
	
	{\em 2019} \\
	Buriticá, J.J., \& \textbf{Alcalá, E.} (2019). Increased Generalization in a Peak Procedure after Delayed Reinforcement. \textit{Behavioural Processes, 169, 103978}.
	
	{\em 2020} \\
	Castiello, S., Burgos, J.E., Buriticá, J., dos Santos, C.V. \& \textbf{Alcalá, E.} (2020). Interacción entre magnitud y probabilidad de reforzamiento en la elección automoldeada. \textit{Revista Mexicana de Análisis de la Conducta.}
	
	Gómez, E. G., García, V. I., Morales, C. S., López, F. A. L., \& \textbf{Alcalá, E.} (2020). \textit{Manual de Análisis de Datos de Descuento Temporal en RStudio (MADDTeR)}. Red Universitaria de Aprendizaje (RUA) de la UNAM. \url{https://www.rua.unam.mx/portal/recursos/ficha/85989}
	
	{\em 2021} \\
	
	Eudave-Patiño, M., \textbf{Alcalá, E.,} Valerio dos Santos, C., Buriticá, J., (2021). Similar attention and performance in female and male CD1 mice in the peak procedure. \textit{Behavioural Processes. In Press}. DOI: \url{10.1016/j.beproc.2021.104443}
	
	López-Cárdenas, P.G, \textbf{Alcalá, E.}, Sánchez-Torres, J.D., Araujo, E. (2021). Enhancing the Sensitivity of a Class of Sensors: A Data-Based Engineering Approach. \textit{2021 IEEE 21st International Conference on Nanotechnology (NANO)}, 221-224, DOI: \url{10.1109/NANO51122.2021.9514352}. 
	
	López-Cárdenas, P. G.,\textbf{ Alcalá, E.}, Sánchez-Torres, J. D., \& Araujo, E. (2021, November). A Resampling Approach for the Data-Based Optimization of Nanosensors. In \textit{2021 18th International Conference on Electrical Engineering, Computing Science and Automatic Control (CCE) (pp. 1-4). IEEE.}
	
	{\em 2022}\\ 
	
	Campos-Ordoñez, T.,\textbf{Alcalá, E.}, Ibarra-Castañeda, N., Buriticá, J., González-Pérez, 0. (2021). A repeated cyclohexane inhalation generates stereotypic circling, hyperlocomotion, persistent anxiety-like behavior, and dysregulates the c-Fos expression in striatum and nucleus accumbens. \textit{Behavioural brain research, 418, 113664}
	
	Sosa, R., \textbf{Alcalá., E}. (2022). The Nervous System as a Solution for Implementing Closed Negative Feedback Control Loops. \textit{Journal of Experimental Analysis of Behavior}, 1-22
	
	% \textbf{Alcalá, E.}, Márquez, I., Lara, E. \& Buriticá, J. (2020). ¿Degradación o elección? Distribución de tiempo entre responder y no responder en un procedimiento de degradación de contingencias. \textit{Manuscrito enviado para publicación}
	
	\end{rSection}

\begin{rSection}{Experiencia}

\begin{rSubsection}{ITESO}{2021}{Posdoctorado}{COECYTJAL - Proyecto FODECYJAL 8248-2019}
% \item Optimización de sensor nanoestructurado para la detección de glucosa.
\item Diseños experimentales, análisis de datos y optimización mediante superficies de respuesta.
\end{rSubsection}

\begin{rSubsection}{ITESO}{2021}{Profesor}{}
    \item Decisiones y Teoría de Juegos en Ingeniería Financiera.
    \item Econometría Básica en la Licenciatura de Gestión Pública y Políticas Globales.
\end{rSubsection}

\begin{rSubsection}{Universidad de la Ciénega}{2019}{Profesor}{Guadalajara, Jal}
\item Clases de bioestadística para la carrera de Nutrición
\end{rSubsection}

\begin{rSubsection}{Consultor ({\em freelancer})}{2018 - }{Análisis de datos y asesoría estadística}{Guadalajara, Jal}
\item Diseño experimental, análisis de datos e inferencia estadística para la toma de decisiones
\item Ejemplo de trabajo: Hospital San Javier, Fistula Day:  \url{https://bit.ly/2Vz2sl7}
\end{rSubsection}

\begin{rSubsection}{UTEG}{2017}{Asesor académico}{Guadalajara, Jal.}
	\item Asesor y tutor de estudiantes talento.
\end{rSubsection}

%------------------------------------------------

\begin{rSubsection}{Instituto Lumiere}{2014 - 2015}{Instructor}{Ocotlán, Jal}
	\item Resolución de problemas de álgebra y cálculo
\end{rSubsection}

\end{rSection}

% Publications


\begin{rSection}{Ponencias y carteles}
{\em 2016} \hfill {\em CUCosta, UdeG, Jalisco} \\
{\bf XXVI Congreso Mexicano de Análisis de la conducta} (Ponencia) \\
Elección automoldeada en redes neurales: sensibilidad a la magnitud y probabilidad de reforzamiento. 

{\em 2017} \hfill {\em UAA, Aguascalientes} \\
{\bf XXVII Congreso Mexicano de Análisis de la conducta} (Ponencia) \\
Impulsividad pavloviana: predicción y prueba de un modelo de redes neurales artificiales. 

{\bf XXVII Congreso Mexicano de Análisis de la conducta} (Ponencia) \\ 
Modelo de redes neurales DDS: herramientas teóricas, predicciones experimentales y una implementación robótica.

{\em 2019} \hfill {\em UNAM-INB, Querétaro} \\
{\bf 2nd Annual Conference of the Timing Research Forum} (Cartel) \\
Increased Generalization after Delayed Reinforcement

{\bf Seminario Internacional sobre Comportamiento y Aplicaciones (SINCA) VIII } (Cartel)
Aplicaciones de aprendizaje estadıstico a estimación temporal con programas de intervalo fijo.\\
\href{https://www.researchgate.net/profile/Emmanuel-Alcala/publication/352672052_Aplicaciones_de_aprendizaje_estadistico_a_estimacion_temporal_con_programas_de_intervalo_fijo/links/60d29f2592851c34e07cdd31/Aplicaciones-de-aprendizaje-estadistico-a-estimacion-temporal-con-programas-de-intervalo-fijo.pdf}{SINCA VIII 2019}

\end{rSection}

\begin{rSection}{Habilidades técnicas {\normalfont (0 - 100 \%)}}

%\begin{tabular}{ @{} >{\bfseries}l @{\hspace{6ex}} l }
%Computer Languages & Prolog, Haskell, AWK, Erlang, Scheme, ML \\
%Protocols \& APIs & XML, JSON, SOAP, REST \\
%Databases & MySQL, PostgreSQL, Microsoft SQL \\
%Tools & SVN, Vim, Emacs
%\end{tabular}

\begin{tabular} { @{} >{\bf}l @{\hspace{6ex}} l }
	\skill{\sf R \& RStudio (IDE)}{70} & Lenguaje, software estadístico y múltiples librerías \\
	\skill{\sf Python 3}{45} & Lenguaje, OpenCV y scikit-learn\\
	\skill{\LaTeXe}{65} & Preparación de documentos científicos técnicos \\
	\skill{Sistemas Operativos}{70} & Windows, Linux (shell scripting básico) y macOS \\
	\skill{Ofimática}{70} & MS Office y LibreOffice\\
	\skill{Inglés}{60} & Escrito y hablado 
\end{tabular}


\end{rSection}

\begin{rSection}{Formación adicional}
	{\em 2016}  \\
	{\bf Teoría de la probabilidad y estadística matemática} \hfill {\em CUCEI, UdG}
	
	{\bf Data Science Bootcamp} \hfill {\em IBM-UdG}

	{\em 2017} \\
	{\bf Model comparison in quantitative analysis of behavior} \hfill {UAA - Randolph Grace, PhD} 
	
	{\bf Curso de Álgebra lineal} \hfill {\em CUCEI, UdG}
	
\end{rSection}

% \begin{rSection}{Acerca de mí}
% Tengo interés en psicología experimental, ciencias cognitivas, neurociencias, estadística y modelación matemática de fenómenos sociales y cognitivos. Por ejemplo, \href{https://gershmanlab.com/index.html}{CCNlab}, o \href{https://nivlab.princeton.edu/}{Niv Lab}, cuya investigación es mi aspiración personal. Después de obtener el grado, decidí estudiar posgrado en Análisis de la Conducta en el Centro de Estudios e Investigaciones en Comportamiento (CEIC), único centro especializado en esos temas en México. Actualmente investigo mecanismos de aprendizaje y modelos de persistencia. Abogo por el software libre, el uso de Linux, Ciencia Abierta y conocimiento libre y reproducibilidad en la ciencia. Tengo un fuerte interés por la filosofía y, fuera de lo académico, por la literatura y cine. 
% \end{rSection}


\end{document}
